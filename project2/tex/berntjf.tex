\documentclass[a4paper,10pt,english]{article}
\usepackage[utf8]{inputenc}
\usepackage[norsk]{babel}
\usepackage{amsmath,graphicx,varioref,verbatim,amsfonts,geometry}
\usepackage[usenames,dvipsnames,svgnames,table]{xcolor}
\usepackage[colorlinks]{hyperref}
\usepackage{pdfpages}
\setlength{\parindent}{0mm}
\setlength{\parskip}{1.5mm}
\usepackage{textcomp}
\usepackage{xcolor}
\usepackage{textcomp}
\usepackage{listings}
\usepackage[backend=biber]{biblatex}
\definecolor{listinggray}{gray}{0.9}
\definecolor{lbcolor}{rgb}{0.9,0.9,0.9}
\usepackage{xparse}
\renewcommand{\thesubsection}{\alph{subsection}}
\usepackage{titling}
\setlength{\droptitle}{-4em}
\addtolength{\hoffset}{-0.5in}
\addtolength{\textwidth}{0.75in}
\addtolength{\voffset}{-.5in}
\addtolength{\textheight}{1.75in}

\NewDocumentCommand{\codeword}{v}{%
\texttt{\textcolor{blue}{#1}}%
}

\lstset{
 backgroundcolor=\color{lbcolor},
 tabsize=4,
 rulecolor=,
 language=matlab,
 basicstyle=\scriptsize,
 upquote=true,
 aboveskip={1.5\baselineskip},
 columns=fixed,
 showstringspaces=false,
 extendedchars=true,
 breaklines=true,
 prebreak = \raisebox{0ex}[0ex][0ex]{\ensuremath{\hookleftarrow}},
 frame=single,
 showtabs=false,
 showspaces=false,
 showstringspaces=false,
 identifierstyle=\ttfamily,
 keywordstyle=\color[rgb]{0,0,1},
 commentstyle=\color[rgb]{0.133,0.545,0.133},
 stringstyle=\color[rgb]{0.627,0.126,0.941},
}

\addbibresource{berntjf.bib}
\title{FYS3150 Project 2
}
\author{Bernt Jonas Fløde
}
\begin{document}
\maketitle



\section*{Sammendrag}

\section*{Introduksjon}
Dette prosjektet skal utvikle en egenverdi-løser basert på Jacobis metode.
Vi begynner med å se på bjelke-i-spenn-problemet, med følgende
differensialligning:
$$\gamma \frac{d^2 u(x)}{dx^2} = -F u(x),$$

\section*{Metode}
\subsection*{Bjelke i spenn}
For å løse differensialligningen brukes tilnærmingen
$$u'' \approx \frac{u(\rho+h) -2u(\rho) +u(\rho-h)}{h^2},$$
hvor $h$ er steget og $\rho = x/L$. Dette er ligningen for en bjelke i spenn,
der $u(x)$ er den vertikale forskyvningen, $F$ er kraften som blir utført på
bjelken og $\gamma$ er en konstant definert utfra bjelken selv.
Definerer så $\rho_i=\rho_0+i h$ og $u_i=u(\rho_i) \hspace{0.5cm} \mathrm{for} \hspace{0.1cm} i=1,2,...,N.$ og slik at
$$h = (\rho_N-\rho_0)/N = 1/N$$
per definisjon av $\rho$, og
$$-\frac{u_{i+1} -2u_i +u_{i-1} }{h^2}\hspace{0.1cm}= \lambda u_i$$
som er et ligningssett med $i+1$ ligninger. På matriseform blir dette
\[
\begin{bmatrix}
     d    & a     & \dots&\dots & 0    \\
     a    & d     & a    & \dots& 0    \\
    \vdots&\ddots &\ddots&\ddots&\vdots\\
     0    &\dots  & a    & d    & a    \\
     0    &\dots  & 0    & a    & d
\end{bmatrix}
\begin{bmatrix}
    u_1 \\ u_2 \\ \vdots \\ u_{N-2} \\ u_{N-1}
\end{bmatrix}
=
\lambda
\begin{bmatrix}
    u_1 \\ u_2 \\ \vdots \\ u_{N-2} \\ u_{N-1}
\end{bmatrix}
\]
der $a = -1/h^2$ og $b = 2/h^2$. Med vår definisjon av $h$ betyr det at
$a = -N^2$ og $b = 2 N^2$.

Dette ligningssettet har de analytiske løsningene
$$\lambda_j = d+2a\cos{(\frac{j\pi}{N+1})} \hspace{0.5cm} \mathrm{for} \hspace{0.1cm} j=1,2,...,N-1.$$

\subsection*{Ortogonalitetens bevaring}
For å vise at en ortogonal transformasjon på ortogonale vektorer bevarer
ortogonaliteten sin, ser vi på en vektorbasis $\mathbf{v}_i$, der
$$\mathbf{v}_i^T \mathbf{v}_i = \delta_{i j} .$$
Ser så på den ortogonale transformasjonensmatrisen $\mathbf{U}$,
$$\mathbf{w}_i = \mathbf{U} \mathbf{v}_i ,$$
og setter inn, der vi bruker egenskapen $\mathbf{U}^T \mathbf{U} = I$ for ortogonale
matriser,
$$\mathbf{w}_i^T \mathbf{w}_i = (\mathbf{U} \mathbf{v}_i)^T \mathbf{U} \mathbf{v}_i = \mathbf{v}_i^T \mathbf{U}^T \mathbf{U} \mathbf{v}_i^T = \mathbf{v}_i^T I \mathbf{v}_i^T = \mathbf{v}_i^T \mathbf{v}_i^T = \delta_{i j},$$
som betyr at ortogonaliteten er bevart.

\subsection*{Metode for å finne egenverdier}
For å finne egenverdier til en matrise kan vi bruke similaritetstransformasjonen $\mathbf{B}=\mathbf{S}^T\mathbf{A}\mathbf{S}$, der $\mathbf{S}$ har egenskapen $\mathbf{S}^{-1}=\mathbf{S}^T$ og $\mathbf{B}$ ideelt skal bli en diagonalmatrise. Hvis $\mathbf{B}$ er en diagonalmatrisene, er elementene på diagonalen egenverdiene til $\mathbf{A}$.

Utfordringen er å finne riktig matrise $\mathbf{S}$. I stedet for å finne helt riktig matrise, blir strategien å bruke similaritetstransformasjon flere ganger, og velge $\mathbf{S}$ slik at vi får en matrise hvor elementene som ikke ligger på diagonalen statig går nærmere 0. Dette kan gjøres med Jacobis metode, som blir forklart i kap. 7.4 i \cite{komp}.

\subsection*{Implementering i C++}
Programmet src/toeplitz.cpp viser i C++ hvordan Jacobis metode kan brukes til å finne løsningene på bjelke-i-spenn-problemet, ved å finne egenverdier og sammenligne dem med den analytiske løsningen.

Slik settes matrisen opp:
\begin{lstlisting}
double* a = new double[N+1];
double* d = new double[N+1];
double N_sq = pow(N, 2.0);
double a_value = -N_sq;
double d_value = 2*N_sq;
for (int i = 0; i < N+1; i++) {
    a[i] = a_value;
    d[i] = d_value;
}
\end{lstlisting}

Med en ekstra variabel $(ih)^2$ til å simulere et elektron i et potensial,
istedenfor fastspent bjelke, settes matrisen opp slik:
\begin{lstlisting}
double* a = new double[N+1];
double* d = new double[N+1];
double N_sq = pow(N, 2.0);
double a_value = -N_sq;
double d_value = 2*N_sq;
for (int i = 0; i < N; i++) {
    a[i] = a_value;
    d[i] = d_value + pow(i+1, 2.0)/N_sq;
}
\end{lstlisting}

Med enda en ekstra variabel $\omega^2 (i h)^2 + 1/(i h)$ for å simulere to
elektrone, settes matrisen opp slik:
\begin{lstlisting}
double* a = new double[N+1];
double* d = new double[N+1];
double N_sq = pow(N, 2.0);
double omega_r_sq = pow(omega_r, 2.0);
double a_value = -N_sq;
double d_value = 2*N_sq;
for (int i = 0; i < N; i++) {
    a[i] = a_value;
    d[i] = d_value + omega_r_sq*pow(i+1, 2.0)/N_sq + N/(i+1.0);
}
\end{lstlisting}

\section*{Resultat og konklusjon}
Det tar 3 iterasjoner for similaritetstransformasjonen å gi en diagonalmatrise uansett om $N=6$, $N=11$ eller $N=101$, noe som antyder at dette er uavhengig av $N$.

\printbibliography
\end{document}
\grid
